\documentclass[fullpage]{article}
\usepackage{graphicx}
\usepackage{enumerate}

\title{\vspace*{-2cm}\bf Assignment 5 - COMP 3400}
\author{Khalil Van Alphen\\
Student Number: 100863992\\
Email: \  khalil.va@gmail.com}
\date{}
\begin{document}
\maketitle
\pagestyle{plain}
\thispagestyle{empty}

\noindent 1. \ [50\%] \ (a) Use Prolog with predicates and variables (not propositional Prolog) to decide
for an {\bf arbitrary finite string} $\varphi$ over the alphabet $\{\wedge, ), \neg, (, \vee, \rightarrow, \leftrightarrow, p, q, r, s, t, ...\}$ (the latter stand for propositional variables) is  a well-formed propositional formula. Use a unary predicate
for this, say ${Decide}(\cdot)$.\\

\noindent {\bf Solution:} \ We will use a boolean algebra with predicate notation to represent our rules. Our set of formula are as follows:
\begin{itemize}
\item {\em $formula(atom(-)).$}
\item {\em $formula(\vee(A, B)):- formula(A), formula(B).$}
\item {\em $formula(\neg(A)):- formula(A).$}
\item {\em $formula(\wedge(A, B)):- formula(A), formula(B).$}
\item {\em $formula(\rightarrow(A, B)):- formula(A), formula(B).$}
\item {\em $formula(\leftrightarrow(A, B)):- formula(A), formula(B).$}
\end{itemize}
Our first base case rule will be applied to each term. Effectively, we must first identify atomic elements in our given formula. By this, we are in a sense declaring that a formula must be a composition of atomic elements recursively, or inductively built from atomic elements. Elements such as ${p,q,r,s,t...}$ are all atomic by definition.\\
\noindent {\bf Examples:} \ 
\begin{itemize}
\item[$\varphi_1$:]~ ${formula(\rightarrow(atom(a),atom(b))).}$ VALID
\item[$\varphi_2$:]~ ${formula(\neg(\wedge(atom(a),\vee(atom(b),atom(c))))).}$ VALID
\item[$\varphi_3$:]~ ${formula(\rightarrow(\neg(atom(a))),atom(b)).}$ VALID
\item[$\varphi_4$:]~ ${formula((atom(a),atom(b))).}$ INVALID
\item[$\varphi_5$:]~ ${formula(\vee(atom(a))).}$ INVALID
\item[$\varphi_6$:]~ ${formula((atom(b),\vee(atom(a),atom(c)))).}$ INVALID
\end{itemize}

\noindent {\bf Traces:} \\
\begin{verbatim}
[trace]  ?- formula(->(atom(a),atom(b))).
   Call: (8) formula((atom(a)->atom(b))) ? creep
   Call: (9) formula(atom(a)) ? creep
   Exit: (9) formula(atom(a)) ? creep
   Call: (9) formula(atom(b)) ? creep
   Exit: (9) formula(atom(b)) ? creep
   Exit: (8) formula((atom(a)->atom(b))) ? creep
true.
[trace]  ?- formula(~(^(atom(a),v(atom(b),atom(c))))).
   Call: (8) formula(~(atom(a)^v(atom(b), atom(c)))) ? creep
   Call: (9) formula(atom(a)^v(atom(b), atom(c))) ? creep
   Call: (10) formula(atom(a)) ? creep
   Exit: (10) formula(atom(a)) ? creep
   Call: (10) formula(v(atom(b), atom(c))) ? creep
   Call: (11) formula(atom(b)) ? creep
   Exit: (11) formula(atom(b)) ? creep
   Call: (11) formula(atom(c)) ? creep
   Exit: (11) formula(atom(c)) ? creep
   Exit: (10) formula(v(atom(b), atom(c))) ? creep
   Exit: (9) formula(atom(a)^v(atom(b), atom(c))) ? creep
   Exit: (8) formula(~(atom(a)^v(atom(b), atom(c)))) ? creep
true.
[trace]  ?- formula(->(~(atom(a)),atom(b))).
   Call: (8) formula((~(atom(a))->atom(b))) ? creep
   Call: (9) formula(~(atom(a))) ? creep
   Call: (10) formula(atom(a)) ? creep
   Exit: (10) formula(atom(a)) ? creep
   Exit: (9) formula(~(atom(a))) ? creep
   Call: (9) formula(atom(b)) ? creep
   Exit: (9) formula(atom(b)) ? creep
   Exit: (8) formula((~(atom(a))->atom(b))) ? creep
true.
[trace]  ?- formula((atom(a),atom(b))).
   Call: (8) formula((atom(a), atom(b))) ? creep
   Fail: (8) formula((atom(a), atom(b))) ? creep
false.
[trace]  ?- formula(v(atom(a))).
   Call: (8) formula(v(atom(a))) ? creep
   Fail: (8) formula(v(atom(a))) ? creep
false.
[trace]  ?- formula((atom(b),v(atom(a),atom(c)))).
   Call: (8) formula((atom(b), v(atom(a), atom(c)))) ? creep
   Fail: (8) formula((atom(b), v(atom(a), atom(c)))) ? creep
false.


\end{verbatim}
An extension to determine the length of a formula is made with the following extra rules:
\begin{itemize}
\item {\em $flength(atom(-), 1).$}
\item {\em $flength(\vee(A, B)):-flength(A, C),flength(B, D),L is C+D.$}
\item {\em $flength(\neg(A)):-flength(A, C)L is C.$}
\item {\em $flength(\wedge(A, B)):-flength(A, C),flength(B, D),L is C+D.$}
\item {\em $flength(\rightarrow(A, B)):-flength(A, C),flength(B, D),L is C+D.$}
\item {\em $flength(\leftrightarrow(A, B)):-flength(A, C),flength(B, D),L is C+D.$}
\end{itemize}
Using the same potential formula from before:
\noindent {\bf Traces:} \\
\begin{verbatim}
[trace]  ?- flength(->(atom(a),atom(b)), L).
   Call: (8) flength((atom(a)->atom(b)), _5646) ? creep
   Call: (9) flength(atom(a), _5884) ? creep
   Exit: (9) flength(atom(a), 1) ? creep
   Call: (9) flength(atom(b), _5884) ? creep
   Exit: (9) flength(atom(b), 1) ? creep
   Call: (9) _5646 is 1+1 ? creep
   Exit: (9) 2 is 1+1 ? creep
   Exit: (8) flength((atom(a)->atom(b)), 2) ? creep
L = 2.
[trace]  ?- flength(~(^(atom(a),v(atom(b),atom(c)))), L).
   Call: (8) flength(~(atom(a)^v(atom(b), atom(c))), _5660) ? creep
   Call: (9) flength(atom(a)^v(atom(b), atom(c)), _5914) ? creep
   Call: (10) flength(atom(a), _5914) ? creep
   Exit: (10) flength(atom(a), 1) ? creep
   Call: (10) flength(v(atom(b), atom(c)), _5914) ? creep
   Call: (11) flength(atom(b), _5914) ? creep
   Exit: (11) flength(atom(b), 1) ? creep
   Call: (11) flength(atom(c), _5914) ? creep
   Exit: (11) flength(atom(c), 1) ? creep
   Call: (11) _5918 is 1+1 ? creep
   Exit: (11) 2 is 1+1 ? creep
   Exit: (10) flength(v(atom(b), atom(c)), 2) ? creep
   Call: (10) _5924 is 1+2 ? creep
   Exit: (10) 3 is 1+2 ? creep
   Exit: (9) flength(atom(a)^v(atom(b), atom(c)), 3) ? creep
   Call: (9) _5660 is 3 ? creep
   Exit: (9) 3 is 3 ? creep
   Exit: (8) flength(~(atom(a)^v(atom(b), atom(c))), 3) ? creep
L = 3.
[trace]  ?- flength(->(~(atom(a)),atom(b)), L).
   Call: (8) flength((~(atom(a))->atom(b)), _5650) ? creep
   Call: (9) flength(~(atom(a)), _5902) ? creep
   Call: (10) flength(atom(a), _5902) ? creep
   Exit: (10) flength(atom(a), 1) ? creep
   Call: (10) _5900 is 1 ? creep
   Exit: (10) 1 is 1 ? creep
   Exit: (9) flength(~(atom(a)), 1) ? creep
   Call: (9) flength(atom(b), _5902) ? creep
   Exit: (9) flength(atom(b), 1) ? creep
   Call: (9) _5650 is 1+1 ? creep
   Exit: (9) 2 is 1+1 ? creep
   Exit: (8) flength((~(atom(a))->atom(b)), 2) ? creep
L = 2.
[trace]  ?- flength((atom(a),atom(b)), L).
   Call: (8) flength((atom(a), atom(b)), _5646) ? creep
   Fail: (8) flength((atom(a), atom(b)), _5646) ? creep
false.
\end{verbatim}

As we can see in the last example, formulas that do not evaluate to well-formed will not have a length under our database of rules.

For our next extension, we can apply the same inductive logic to construct from scratch a list of variables use in the formula. We know that an atomic formula has only one variable, so the list can be created with that as a base case. We make use of the built in $append/3$ operation to be efficient:
\begin{itemize}
\item {\em $lvar(atom(A), [A]).$}
\item {\em $lvar(\vee(A, B), L):-lvar(A, C),lvar(B, D),append(C, D, L).$}
\item {\em $lvar(\neg(A), L):-lvar(A, L).$}
\item {\em $lvar(\wedge(A, B), L):-lvar(A, C),lvar(B, D),append(C, D, L).$}
\item {\em $lvar(\rightarrow(A, B), L):-lvar(A, C),lvar(B, D),append(C, D, L).$}
\item {\em $\leftrightarrow(<->(A, B), L):-lvar(A, C),lvar(B, D),append(C, D, L).$}
\end{itemize}
\noindent {\bf Traces:}
\begin{verbatim}
[trace]  ?- lvar(->(atom(a),atom(b)), L).
   Call: (8) lvar((atom(a)->atom(b)), _5646) ? creep
   Call: (9) lvar(atom(a), _5882) ? creep
   Exit: (9) lvar(atom(a), [a]) ? creep
   Call: (9) lvar(atom(b), _5888) ? creep
   Exit: (9) lvar(atom(b), [b]) ? creep
   Call: (9) lists:append([a], [b], _5646) ? creep
   Exit: (9) lists:append([a], [b], [a, b]) ? creep
   Exit: (8) lvar((atom(a)->atom(b)), [a, b]) ? creep
L = [a, b].
[trace]  ?- lvar(~(^(atom(a),v(atom(b),atom(c)))), L).
   Call: (8) lvar(~(atom(a)^v(atom(b), atom(c))), _5660) ? creep
   Call: (9) lvar(atom(a)^v(atom(b), atom(c)), _5660) ? creep
   Call: (10) lvar(atom(a), _5914) ? creep
   Exit: (10) lvar(atom(a), [a]) ? creep
   Call: (10) lvar(v(atom(b), atom(c)), _5920) ? creep
   Call: (11) lvar(atom(b), _5920) ? creep
   Exit: (11) lvar(atom(b), [b]) ? creep
   Call: (11) lvar(atom(c), _5926) ? creep
   Exit: (11) lvar(atom(c), [c]) ? creep
   Call: (11) lists:append([b], [c], _5934) ? creep
   Exit: (11) lists:append([b], [c], [b, c]) ? creep
   Exit: (10) lvar(v(atom(b), atom(c)), [b, c]) ? creep
   Call: (10) lists:append([a], [b, c], _5660) ? creep
   Exit: (10) lists:append([a], [b, c], [a, b, c]) ? creep
   Exit: (9) lvar(atom(a)^v(atom(b), atom(c)), [a, b, c]) ? creep
   Exit: (8) lvar(~(atom(a)^v(atom(b), atom(c))), [a, b, c]) ? creep
L = [a, b, c].
Ԁ[trace]  ?- lvar(->(~(atom(a)),atom(b)), L).
   Call: (8) lvar((~(atom(a))->atom(b)), _5710) ? creep
   Call: (9) lvar(~(atom(a)), _5962) ? creep
   Call: (10) lvar(atom(a), _5962) ? creep
   Exit: (10) lvar(atom(a), [a]) ? creep
   Exit: (9) lvar(~(atom(a)), [a]) ? creep
   Call: (9) lvar(atom(b), _5968) ? creep
   Exit: (9) lvar(atom(b), [b]) ? creep
   Call: (9) lists:append([a], [b], _5710) ? creep
   Exit: (9) lists:append([a], [b], [a, b]) ? creep
   Exit: (8) lvar((~(atom(a))->atom(b)), [a, b]) ? creep
L = [a, b].
\end{verbatim}

\newpage


\noindent 2. \ [50\%]Define lists, i.e. a general predicate ${list}(\cdot)$ that is true with (finite) lists  over the alphabet $A = \{a,b,c,d,e, f, g, h\}$. For example, the list $bdeh$, with first element $b$ and last $h$, should be represented by the term $\nit{cons(b,cons(d,cons(e,cons(h,nil))))}$, and ${list}(\nit{cons(b,cons(d,cons(e,}{cons(h,nil))))})$ should be true. Use Prolog to verify that $\nit{cons(b,cons(d,}{cons(e,cons(h,nil))))}$ is a list, but not $\nit{cons(b,cons(d,e),cons(h,nil))))}$.\\\\
Define a binary predicate $\nit{sublist}(\cdot,\cdot)$, such  that $\nit{sublist}(L_1,L_2)$ becomes true when $L_1$ is a sublist of $L_2$.\\\\ Use Prolog to verify (with the representation in (a)) that 
$ab$ is a sublist of $efabde$. But $fd$ is not a sublist of the latter. 
Use the representation in (a) to define the length of a list. It should be $\nit{length}(efabde,6)$ true. Use Prolog to verify this. \ (You can use Prolog's built-in numbers and arithmetic.)\\\\
\noindent {\bf Solution:} \
\begin{itemize}
\item {\em $list(cons(-, nil)).$}
\item {\em $list(cons(-, B)):-list(B).$}
\end{itemize}
The above rule set defined a list inductively. The first formula is our base case for a list; one that is a single item appended with the empty list. From this, any list item attached to the begin creates a valid list.\\
\noindent {\bf Examples:} \
\begin{itemize}
\item[$\varphi_1$:]~ $cons(a,cons(b,cons(c,nil))).$ VALID
\item[$\varphi_2$:]~ $cons(b,cons(c,cons(d,cons(h,nil))))$ VALID
\item[$\varphi_3$:]~ $cons(cons(a,nil)).$ INVALID
\item[$\varphi_4$:]~ $cons(b,cons(d,e),cons(h,nil))))$ INVALID
\end{itemize}
\noindent {\bf Traces:} \
\begin{verbatim}
[trace]  ?- list(cons(a,cons(b,(cons(c,nil))))).
   Call: (8) list(cons(a, cons(b, cons(c, nil)))) ? creep
   Call: (9) list(cons(b, cons(c, nil))) ? creep
   Call: (10) list(cons(c, nil)) ? creep
   Exit: (10) list(cons(c, nil)) ? creep
   Exit: (9) list(cons(b, cons(c, nil))) ? creep
   Exit: (8) list(cons(a, cons(b, cons(c, nil)))) ? creep
true .
[trace]  ?- list(cons(b,cons(c,cons(d,cons(h,nil))))).
   Call: (8) list(cons(b, cons(c, cons(d, cons(h, nil))))) ? creep
   Call: (9) list(cons(c, cons(d, cons(h, nil)))) ? creep
   Call: (10) list(cons(d, cons(h, nil))) ? creep
   Call: (11) list(cons(h, nil)) ? creep
   Exit: (11) list(cons(h, nil)) ? creep
   Exit: (10) list(cons(d, cons(h, nil))) ? creep
   Exit: (9) list(cons(c, cons(d, cons(h, nil)))) ? creep
   Exit: (8) list(cons(b, cons(c, cons(d, cons(h, nil))))) ? creep
true .
[trace]  ?- list(cons(cons(a,nil))).
   Call: (8) list(cons(cons(a, nil))) ? creep
   Fail: (8) list(cons(cons(a, nil))) ? creep
false.
[trace]  ?- list(cons(b,cons(d,e),cons(h,nil))).
   Call: (8) list(cons(b, cons(d, e), cons(h, nil))) ? creep
   Fail: (8) list(cons(b, cons(d, e), cons(h, nil))) ? creep
false.

\end{verbatim}
As an extension, we can define $sublist(L1,L2)$ to determine whether a list is a sublist (L1 exists somewhere in L2). The logic behind this is we check for each level of the list whether the rest of the list is a prefix of the goal list. To navigate through the list we can use the suffix function which cuts the head of the list off.
\begin{itemize}
\item {\em $prefix(A, L) :- consAppend(A, _, L).$}
\item {\em $suffix(B, L) :- consAppend(_, B, L).$}
\item {\em $sublist(L1, L2) :- suffix(S, L2), prefix(L1, S).$}
\end{itemize}
\noindent {\bf Examples:} \
\begin{itemize}
\item[$\varphi_1$:]~ $sublist(cons(1,cons(2,nil)), cons(1,cons(2,cons(3,nil)))).$
\item[$\varphi_2$:]~ $sublist(cons(4,cons(5,nil)),cons(1,cons(2,cons(3,cons(4,cons(5,nil)))))).$
\end{itemize}
\noindent {\bf Traces:} \
\begin{verbatim}
[trace]  ?- sublist(cons(1,cons(2,nil)), cons(1,cons(2,cons(3,nil)))).
   Call: (8) sublist(cons(1, cons(2, nil)), cons(1, cons(2, cons(3, nil)))) ? creep
   Call: (9) suffix(_5884, cons(1, cons(2, cons(3, nil)))) ? creep
   Call: (10) consAppend(_5884, _5886, cons(1, cons(2, cons(3, nil)))) ? creep
   Exit: (10) consAppend(nil, cons(1, cons(2, cons(3, nil))), cons(1, cons(2, cons(3, nil)))) ? creep
   Exit: (9) suffix(cons(1, cons(2, cons(3, nil))), cons(1, cons(2, cons(3, nil)))) ? creep
   Call: (9) prefix(cons(1, cons(2, nil)), cons(1, cons(2, cons(3, nil)))) ? creep
   Call: (10) consAppend(cons(1, cons(2, nil)), _5886, cons(1, cons(2, cons(3, nil)))) ? creep
   Call: (11) consAppend(cons(2, nil), _5886, cons(2, cons(3, nil))) ? creep
   Call: (12) consAppend(nil, _5886, cons(3, nil)) ? creep
   Exit: (12) consAppend(nil, cons(3, nil), cons(3, nil)) ? creep
   Exit: (11) consAppend(cons(2, nil), cons(3, nil), cons(2, cons(3, nil))) ? creep
   Exit: (10) consAppend(cons(1, cons(2, nil)), cons(3, nil), cons(1, cons(2, cons(3, nil)))) ? creep
   Exit: (9) prefix(cons(1, cons(2, nil)), cons(1, cons(2, cons(3, nil)))) ? creep
   Exit: (8) sublist(cons(1, cons(2, nil)), cons(1, cons(2, cons(3, nil)))) ? creep
true .
[trace]  ?- sublist(cons(4,cons(5,nil)),cons(1,cons(2,cons(3,cons(4,cons(5,nil)))))).
   Call: (8) sublist(cons(4, cons(5, nil)), cons(1, cons(2, cons(3, cons(4, cons(5, nil)))))) ? creep
   Call: (9) suffix(_5926, cons(1, cons(2, cons(3, cons(4, cons(5, nil)))))) ? creep
   Call: (10) consAppend(_5926, _5928, cons(1, cons(2, cons(3, cons(4, cons(5, nil)))))) ? creep
   Exit: (10) consAppend(nil, cons(1, cons(2, cons(3, cons(4, cons(5, nil))))), cons(1, cons(2, cons(3, cons(4, cons(5, nil)))))) ? creep
   Exit: (9) suffix(cons(1, cons(2, cons(3, cons(4, cons(5, nil))))), cons(1, cons(2, cons(3, cons(4, cons(5, nil)))))) ? creep
   Call: (9) prefix(cons(4, cons(5, nil)), cons(1, cons(2, cons(3, cons(4, cons(5, nil)))))) ? creep
   Call: (10) consAppend(cons(4, cons(5, nil)), _5928, cons(1, cons(2, cons(3, cons(4, cons(5, nil)))))) ? creep
   Fail: (10) consAppend(cons(4, cons(5, nil)), _5928, cons(1, cons(2, cons(3, cons(4, cons(5, nil)))))) ? creep
   Fail: (9) prefix(cons(4, cons(5, nil)), cons(1, cons(2, cons(3, cons(4, cons(5, nil)))))) ? creep
   Redo: (10) consAppend(_5926, _5928, cons(1, cons(2, cons(3, cons(4, cons(5, nil)))))) ? creep
   Call: (11) consAppend(_5914, _5934, cons(2, cons(3, cons(4, cons(5, nil))))) ? creep
   Exit: (11) consAppend(nil, cons(2, cons(3, cons(4, cons(5, nil)))), cons(2, cons(3, cons(4, cons(5, nil))))) ? creep
   Exit: (10) consAppend(cons(1, nil), cons(2, cons(3, cons(4, cons(5, nil)))), cons(1, cons(2, cons(3, cons(4, cons(5, nil)))))) ? creep
   Exit: (9) suffix(cons(2, cons(3, cons(4, cons(5, nil)))), cons(1, cons(2, cons(3, cons(4, cons(5, nil)))))) ? creep
   Call: (9) prefix(cons(4, cons(5, nil)), cons(2, cons(3, cons(4, cons(5, nil))))) ? creep
   Call: (10) consAppend(cons(4, cons(5, nil)), _5934, cons(2, cons(3, cons(4, cons(5, nil))))) ? creep
   Fail: (10) consAppend(cons(4, cons(5, nil)), _5934, cons(2, cons(3, cons(4, cons(5, nil))))) ? creep
   Fail: (9) prefix(cons(4, cons(5, nil)), cons(2, cons(3, cons(4, cons(5, nil))))) ? creep
   Redo: (11) consAppend(_5914, _5934, cons(2, cons(3, cons(4, cons(5, nil))))) ? creep
   Call: (12) consAppend(_5920, _5940, cons(3, cons(4, cons(5, nil)))) ? creep
   Exit: (12) consAppend(nil, cons(3, cons(4, cons(5, nil))), cons(3, cons(4, cons(5, nil)))) ? creep
   Exit: (11) consAppend(cons(2, nil), cons(3, cons(4, cons(5, nil))), cons(2, cons(3, cons(4, cons(5, nil))))) ? creep
   Exit: (10) consAppend(cons(1, cons(2, nil)), cons(3, cons(4, cons(5, nil))), cons(1, cons(2, cons(3, cons(4, cons(5, nil)))))) ? creep
   Exit: (9) suffix(cons(3, cons(4, cons(5, nil))), cons(1, cons(2, cons(3, cons(4, cons(5, nil)))))) ? creep
   Call: (9) prefix(cons(4, cons(5, nil)), cons(3, cons(4, cons(5, nil)))) ? creep
   Call: (10) consAppend(cons(4, cons(5, nil)), _5940, cons(3, cons(4, cons(5, nil)))) ? creep
   Fail: (10) consAppend(cons(4, cons(5, nil)), _5940, cons(3, cons(4, cons(5, nil)))) ? creep
   Fail: (9) prefix(cons(4, cons(5, nil)), cons(3, cons(4, cons(5, nil)))) ? creep
   Redo: (12) consAppend(_5920, _5940, cons(3, cons(4, cons(5, nil)))) ? creep
   Call: (13) consAppend(_5926, _5946, cons(4, cons(5, nil))) ? creep
   Exit: (13) consAppend(nil, cons(4, cons(5, nil)), cons(4, cons(5, nil))) ? creep
   Exit: (12) consAppend(cons(3, nil), cons(4, cons(5, nil)), cons(3, cons(4, cons(5, nil)))) ? creep
   Exit: (11) consAppend(cons(2, cons(3, nil)), cons(4, cons(5, nil)), cons(2, cons(3, cons(4, cons(5, nil))))) ? creep
   Exit: (10) consAppend(cons(1, cons(2, cons(3, nil))), cons(4, cons(5, nil)), cons(1, cons(2, cons(3, cons(4, cons(5, nil)))))) ? creep
   Exit: (9) suffix(cons(4, cons(5, nil)), cons(1, cons(2, cons(3, cons(4, cons(5, nil)))))) ? creep
   Call: (9) prefix(cons(4, cons(5, nil)), cons(4, cons(5, nil))) ? creep
   Call: (10) consAppend(cons(4, cons(5, nil)), _5946, cons(4, cons(5, nil))) ? creep
   Call: (11) consAppend(cons(5, nil), _5946, cons(5, nil)) ? creep
   Call: (12) consAppend(nil, _5946, nil) ? creep
   Exit: (12) consAppend(nil, nil, nil) ? creep
   Exit: (11) consAppend(cons(5, nil), nil, cons(5, nil)) ? creep
   Exit: (10) consAppend(cons(4, cons(5, nil)), nil, cons(4, cons(5, nil))) ? creep
   Exit: (9) prefix(cons(4, cons(5, nil)), cons(4, cons(5, nil))) ? creep
   Exit: (8) sublist(cons(4, cons(5, nil)), cons(1, cons(2, cons(3, cons(4, cons(5, nil)))))) ? creep
true .

\end{verbatim}

Lastly, we can define $lilength(list,size)$ to check the size of a list:
\begin{itemize}
\item {\em $lilength(cons(_,nil),1).$}
\item {\em $lilength(cons(A,B), L):-lilength(B, D), L is 1+D.$}
\end{itemize}
\noindent {\bf Examples:} \
\begin{itemize}
\item[$\varphi_1$:]~ $cons(1,cons(2,cons(3,cons(4,nil))))$ = SIZE 4
\item[$\varphi_2$:]~ $cons(4,cons(5,cons(5,cons(4,(cons(3,nil))))))$ = SIZE 5
\end{itemize}
\noindent {\bf Traces:} \
\begin{verbatim}
[trace]  ?- lilength(cons(1,cons(2,cons(3,cons(4,nil)))), 4).
   Call: (8) lilength(cons(1, cons(2, cons(3, cons(4, nil)))), 4) ? creep
   Call: (9) lilength(cons(2, cons(3, cons(4, nil))), _6308) ? creep
   Call: (10) lilength(cons(3, cons(4, nil)), _6308) ? creep
   Call: (11) lilength(cons(4, nil), _6308) ? creep
   Exit: (11) lilength(cons(4, nil), 1) ? creep
   Call: (11) _6312 is 1+1 ? creep
   Exit: (11) 2 is 1+1 ? creep
   Exit: (10) lilength(cons(3, cons(4, nil)), 2) ? creep
   Call: (10) _6318 is 1+2 ? creep
   Exit: (10) 3 is 1+2 ? creep
   Exit: (9) lilength(cons(2, cons(3, cons(4, nil))), 3) ? creep
   Call: (9) 4 is 1+3 ? creep
   Exit: (9) 4 is 1+3 ? creep
   Exit: (8) lilength(cons(1, cons(2, cons(3, cons(4, nil)))), 4) ? creep
true .
[trace]  ?- lilength(cons(4,cons(5,cons(5,cons(4,(cons(3,nil)))))), 5).
   Call: (8) lilength(cons(4, cons(5, cons(5, cons(4, cons(3, nil))))), 5) ? creep
   Call: (9) lilength(cons(5, cons(5, cons(4, cons(3, nil)))), _6374) ? creep
   Call: (10) lilength(cons(5, cons(4, cons(3, nil))), _6374) ? creep
   Call: (11) lilength(cons(4, cons(3, nil)), _6374) ? creep
   Call: (12) lilength(cons(3, nil), _6374) ? creep
   Exit: (12) lilength(cons(3, nil), 1) ? creep
   Call: (12) _6378 is 1+1 ? creep
   Exit: (12) 2 is 1+1 ? creep
   Exit: (11) lilength(cons(4, cons(3, nil)), 2) ? creep
   Call: (11) _6384 is 1+2 ? creep
   Exit: (11) 3 is 1+2 ? creep
   Exit: (10) lilength(cons(5, cons(4, cons(3, nil))), 3) ? creep
   Call: (10) _6390 is 1+3 ? creep
   Exit: (10) 4 is 1+3 ? creep
   Exit: (9) lilength(cons(5, cons(5, cons(4, cons(3, nil)))), 4) ? creep
   Call: (9) 5 is 1+4 ? creep
   Exit: (9) 5 is 1+4 ? creep
   Exit: (8) lilength(cons(4, cons(5, cons(5, cons(4, cons(3, nil))))), 5) ? creep
true .
\end{verbatim}

\newpage
\noindent {\bf Appendices I:} \\
Below is the full .pl file used for testing. It contains the full definitions and facts as defined by the solutions above.
\begin{verbatim}
formula(atom(_)).
formula(v(A, B)):- formula(A), formula(B).
formula(~(A)):- formula(A).
formula(^(A, B)):- formula(A), formula(B).
formula(->(A, B)):- formula(A), formula(B).
formula(<->(A, B)):- formula(A), formula(B).

flength(atom(_), 1).
flength(v(A, B), L):-flength(A, C),flength(B, D),L is C+D.
flength(~(A), L):-flength(A, C),L is C.
flength(^(A, B), L):-flength(A, C),flength(B, D),L is C+D.
flength(->(A, B), L):-flength(A, C),flength(B, D),L is C+D.
flength(<->(A, B), L):-flength(A, C),flength(B, D),L is C+D.

lvar(atom(A), [A]).
lvar(v(A, B), L):-lvar(A, C),lvar(B, D),append(C, D, L).
lvar(~(A), L):-lvar(A, L).
lvar(^(A, B), L):-lvar(A, C),lvar(B, D),append(C, D, L).
lvar(->(A, B), L):-lvar(A, C),lvar(B, D),append(C, D, L).
lvar(<->(A, B), L):-lvar(A, C),lvar(B, D),append(C, D, L).

list(cons(_, nil)).
list(cons(_, B)):-list(B).

consAppend(nil,X,X).
consAppend(cons(X,L1),L2,cons(X,L3)):-consAppend(L1,L2,L3).
prefix(A, L) :- consAppend(A, _, L).
suffix(B, L) :- consAppend(_, B, L).
sublist(L1, L2) :- suffix(S, L2), prefix(L1, S).

lilength(cons(_,nil),1).
lilength(cons(A,B), L):-lilength(B, D), L is 1+D.
\end{verbatim}
\end{document} 

